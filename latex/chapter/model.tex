\chapter{Models}

In this thesis there are multiple "models" and we will therefor establish cleare notations and descriptions of 
each.

\section{\gls{ai-model} - the AI model to be learned}

With $\Theta_{ai}$ we refere to the AI model we are trying to teach to a human. 
In this thesis we will experimented with two different implementations of $\theta{ai}$. 
The first being \gls{CNN} \cite{CNN:deeplearning}, and the other being \gls{FCNN}.
$\Theta_{ai}$ will be trained on imags, and will try to preditc a ground truth boolean function. 
This will be discussed in the chapter \nameref{DatasetChap}. Both \gls{CNN} and \gls{FCNN} will be discussed in further implementations detail in [chapter FCNN] and [chapter CNN].

\section{$L_M$ - modeling human learner}
$L_M$ will be an algorithm trying to mimic human behavior. Given a set of \hyperref[InstanceDescription]{instances} and corresponding predicted values, $L_M$ 
outputs a guessed $\Theta_{M}$. \{Should i here discuss different implementations, difficulty of mimicing human behavior etc?\}

\section{$\Theta_{M}$ - $L_M$s model of $\Theta_{AI}$}
$\Theta_{M}$ is the output of $L_M$, and a model of $\Theta_{AI}$. Model is understood as giving the same output on, essentially[note to explaination], the same input.
Our $\Theta_{M}$ will be a boolean function, and hence not take in a bitmap. Rather it takes as input a truth assigment to four literals, 
corresponding to the four litterals "A", "B", "C", and "D". 
