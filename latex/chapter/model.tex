\chapter{Model}

In this thesis there are multiple "models" and we will therefor establish cleare notations and descriptions of 
each.

\section{$\theta_{ai}$ - the AI model to be learned}

With $\theta_{ai}$ we refere to the AI model we are trying to teach to a human. 
In this thesis we will experimented with two different implementations of $\theta{ai}$. 
The first being \gls{CNN} \cite{CNN:deeplearning}, and the other being \gls{FCNN}.
$\theta_{ai}$ will be trained on imags, and will try to preditc a ground truth boolean function. 
This will be discussed in the chapter \nameref{DatasetChap}. Both \gls{CNN} and \gls{FCNN} will be discussed in further implementations detail.

\section{$\theta_{LM}$ - modeling human learner}
$\theta_{LM}$ will be an algorithm trying to mimic human behavior. Given a set of \hyperref[InstanceDescription]{instances} and corresponds predicted values $\theta_{LM}$ 
outputs a guessed $\theta_{ai}$ it belives made these prediction for the given instances.